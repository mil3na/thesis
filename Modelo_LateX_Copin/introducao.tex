\chapter{Introdu\c{c}\~{a}o}

\begin{itemize}
\item apresentação geral do assunto do trabalho;
\item definição sucinta e objetivo do tema abordado;
\item justificativa sobre a escolha do tema e métodos empregados;
\item delimitação precisa das fronteiras da pesquisa em relação ao campo e períodos abrangidos;
\item esclarecimentos sobre o ponto de vista sob o qual o assunto será tratado;
\item relacionamento do trabalho com outros da mesma área;
\item objetivos e finalidades da pesquisa, com especificação dos aspectos que serão ou não abordados;
\item a proposição poderá ser apresentada em capítulo à parte. 
\end{itemize}

\begin{itemize}
\item Vamos estudar aqueles usuários os quais seus colegas podem fácilmente inferir o gênero;
\item O que podemos aprender fazendo estas comparações?
\item Por que os posts das mulheres seria diferente dos feitos por homens?

\end{itemize}

Mulheres tendem a se identificar menos, mas isso não é relevante, já que nós queremos ver o impacto que "ser mulher" nestas comunidades tem na participação destes usuários.