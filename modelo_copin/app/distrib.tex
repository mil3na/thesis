%!TEX root = ../main_mestrado.tex
\chapter{Cinco maiores comunidades}
\label{app:distrib}

\begin{figure}
  \centering
  \includegraphics[height=0.65\paperheight]{figures/stackoverflow.pdf}
  \caption[Distribuição das variáveis para o site \emph{StackOverflow}]{Distribuição das variáveis onde foi aplicado o teste \emph{Mann-Whitney-U} para o site \emph{StackOverflow}. Mostramos um histograma do log do valor da variável mais um para as variáveis discretas e a gráfico da função de densidade, para as variáveis contínuas. }
\end{figure}

\begin{figure}
	\centering
  \includegraphics[height=0.65\paperheight]{figures/superuser.pdf}
 \caption[Distribuição das variáveis para o site \emph{SuperUser}]{Distribuição das variáveis onde foi aplicado o teste \emph{Mann-Whitney-U} para o site \emph{SuperUser}. Mostramos um histograma do log do valor da variável mais um para as variáveis discretas e a gráfico da função de densidade, para as variáveis contínuas. }
\end{figure}

\begin{figure}
	\centering
  \includegraphics[height=0.65\paperheight]{figures/serverfault.pdf}
  \caption[Distribuição das variáveis para o site \emph{ServerFault}]{Distribuição das variáveis onde foi aplicado o teste \emph{Mann-Whitney-U} para o site \emph{ServerFault}. Mostramos um histograma do log do valor da variável mais um para as variáveis discretas e a gráfico da função de densidade, para as variáveis contínuas. }
\end{figure}

\begin{figure}
	\centering
  \includegraphics[height=0.65\paperheight]{figures/math.pdf}
  \caption[Distribuição das variáveis para o site \emph{Mathematics}]{Distribuição das variáveis onde foi aplicado o teste \emph{Mann-Whitney-U} para o site \emph{Mathematics}. Mostramos um histograma do log do valor da variável mais um para as variáveis discretas e a gráfico da função de densidade, para as variáveis contínuas. }
\end{figure}

\begin{figure}
  \centering
  \includegraphics[height=0.65\paperheight]{figures/programmers.pdf}
  \caption[Distribuição das variáveis para o site \emph{Programmers}]{Distribuição das variáveis onde foi aplicado o teste \emph{Mann-Whitney-U} para o site \emph{Programmers}. Mostramos um histograma do log do valor da variável mais um para as variáveis discretas e a gráfico da função de densidade, para as variáveis contínuas. }
\end{figure}
