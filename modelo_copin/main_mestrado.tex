%%%%%%%%%%%%%%%%%%%%%%%%%%%%%%%%%%%%%%%%%%%%%%%%%%%%%%%%%%%%%%%%%%%%%%%%%%%%%%%%
%%
%% Para utilizar ese modelo sao necessarios os seguintes arquivos:
%%
%% copin.cls
%% copin.sty
%% mestre.sty
%%
%%%%%%%%%%%%%%%%%%%%%%%%%%%%%%%%%%%%%%%%%%%%%%%%%%%%%%%%%%%%%%%%%%%%%%%%%%%%%%%%

\documentclass[a4paper,titlepage]{copin}
\usepackage[portuges,english]{babel}
\usepackage{copin,mestre,epsfig}
\usepackage{times}

%-------------------------- Para usar acentuacaoo em sistemas ISO8859-1 ------------------------------------
% Se estiver usando o Microsoft Windows ou linux com essa codificacao, descomente essa linhas abaixo
% e comente as linhas referentes ao UTF8
% \usepackage[latin1]{inputenc} % Usar acentuacao em sistemas ISO8859-1, comentar a linha com  \usepackage[utf8x]{inputenc}
%-----------------------------------------------------------------------------------------------------

%-------------------------- Para usar acentuacao em sistemas UTF8 ------------------------------------
% Para a maior parte das distribuicoes linux, usar a opcao utf8x (lembrar de comentar as linha referente a ISO8859-1 acima)
\usepackage{ucs}
\usepackage[utf8x]{inputenc}
% \usepackage[utf8]{inputenc}
\usepackage[T1]{fontenc}
%-----------------------------------------------------------------------------------------------------


\usepackage{fancyheadings}
\usepackage{url} 
\usepackage{graphicx}
\usepackage{longtable} %tabelas longas, para tabelas que ultrapassam uma pagina
%\input{psfig.sty}

\usepackage{balance}  % to better equalize the last page
\usepackage{graphics} % for EPS, load graphicx instead 
\usepackage{txfonts}
\usepackage{times}    % comment if you want LaTeX's default font
\usepackage[pdftex]{hyperref}
\usepackage{url}      % llt: nicely formatted URLs
\usepackage{color}
\usepackage{textcomp}
\usepackage{booktabs}
\usepackage{ccicons}
\usepackage{todonotes}
\usepackage{graphicx}
\usepackage{tabularx}


% ----------------- Para inserir codigo fonte de linguagens de programacao no documento -------------
\usepackage{listings}
\lstset{numbers=left,
stepnumber=1,
firstnumber=1,
%numberstyle=\tiny,
extendedchars=true,
breaklines=true,
frame=tb,
basicstyle=\footnotesize,
stringstyle=\ttfamily,
showstringspaces=false
}
\renewcommand{\lstlistingname}{C\'odigo Fonte}
\renewcommand{\lstlistlistingname}{Lista de C\'odigos Fonte}
% ---------------------------------------------------------------------------------------------------

\selectlanguage{portuges}
\sloppy



\begin{document}



%%%%%%%%%%%%%%%%%%%%%%%%%%%%%%%%%%%%%%%%%%%%%%%%%%%%%%%%%%%%%%%%%%%%%%%%%%%%%%%%
% \Titulo{Desmistificando Estereótipos de Gênero em Comunidades de Perguntas e Respostas}
% \Titulo{Diferenças (e similaridades) entre gêneros quanto às suas contribuições a sites de Perguntas e Respostas}
\Titulo{?}
% Gender differences (and similarities) in contribution behavior on social Q\&A sites
\Autor{Milena Sales Araujo}
\Data{30/06/2015}
\Area{Ciência da Computação}
\Pesquisa{Linha de Pesquisa}
\Orientadores{Nazareno Ferreira de Andrade  \\
	 (Orientador)}

\newpage
\cleardoublepage

\PaginadeRosto

\newpage
\cleardoublepage

%%%%%%%%%%%%%%%%%%%%%%%%%%%%%%%%%%%%%%%%%%%%%%%%%%%%%%%%%%%%%%%%%%%%%%%%%%%%%%%%
\begin{resumo} 
%!TEX root = main_mestrado.tex
Sites de perguntas e respostas são conhecidos por serem uma fonte de conhecimento simples e confiável. Os sites do StackExchange se destacam estando dentre as maiores comunidades de perguntas e respostas online, principalmente se considerarmos aquelas relacionados à STEM. Assim como em outras comunidades online, mulheres são uma minoria nos sites do Stackexchange, fazendo com que estes sites não se beneficiem de pontos de vista diversificados na criação de seu conteúdo. Esta pesquisa investiga se mulheres que se dedicam aos sites do StackExchange são influenciadas a contribuir menos e deixar a comunidade. Para isso, examinamos diferenças entre os gêneros quanto ao número de contribuições feitas aos sites, o tempo que passaram contribuindo e quanto a diferença de qualidade das suas contribuições segundo a comunidade. Nossos resultados mostram que, na maioria dos sites, mulheres contribuem tanto quanto homens, sendo suas contribuições com qualidade e por intervalos de tempo similares aos homens. Quando percebe-se diferença no número de contribuições, são as mulheres que tendem a se destacar. Mais ainda, encontramos que a proporção de contribuições vindas de mulheres nos sites relacionados a STEM do StackExchange estão aumentando em algumas comunidades. Estes resultados ajudam a melhorar a visão que temos das contribuições feitas por mulheres neste tipo de site, gerando repercussão para pesquisas futuras assim como discussão de design para os administradores.
\end{resumo}

\newpage
\cleardoublepage

%%%%%%%%%%%%%%%%%%%%%%%%%%%%%%%%%%%%%%%%%%%%%%%%%%%%%%%%%%%%%%%%%%%%%%%%%%%%%%%%
\begin{summary}
Question and answer communities are known as a simple and reliable knowledge database. StackExchange group is standing out as one of the biggest of its kind, particularly the ones related to STEM. This work investigates how the amount contributed to such communities, the time spent contributing, contributions over time and the amount of registrations over time are different for contributors that are men or women, as well as whether the quality of such contributions is perceived as equal by the community. Our results point that overall women who are contributing contribute as much, with similar quality, and for a similar period as men. There are however noticeable difference in the preferred type of contribution: women tend to ask more, while men are more likely to contribute answers. These results point to a richer picture of female contribution in those sites, and raise several implications for future research, as well as to site administrators.
\end{summary}

\newpage
\cleardoublepage

%%%%%%%%%%%%%%%%%%%%%%%%%%%%%%%%%%%%%%%%%%%%%%%%%%%%%%%%%%%%%%%%%%%%%%%%%%%%%%%%
\begin{agradecimentos}
Gostaria de agradecer à comunidade StackExchange por ser incrível e pelos dados fornecidos para esta pesquisa. Em especial, agradeço à comunidade Cross Validated e o usuário \textit{Glen\_b} que me deu suporte técnico e estatístico para variadas questões que surgiram durante esta pesquisa. 
\end{agradecimentos}

\clearpage

%%%%%%%%%%%%%%%%%%%%%%%%%%%%%%%%%%%%%%%%%%%%%%%%%%%%%%%%%%%%%%%%%%%%%%%%%%%%%%%%
%% Definicao do cabecalho: secao do lado esquerdo e numero da pagina do lado direito
\pagestyle{fancy}
\addtolength{\headwidth}{\marginparsep}\addtolength{\headwidth}{\marginparwidth}\headwidth = \textwidth
\renewcommand{\chaptermark}[1]{\markboth{#1}{}}
\renewcommand{\sectionmark}[1]{\markright{\thesection\ #1}}\lhead[\fancyplain{}{\bfseries\thepage}]%
	     {\fancyplain{}{\emph{\rightmark}}}\rhead[\fancyplain{}{\bfseries\leftmark}]%
             {\fancyplain{}{\bfseries\thepage}}\cfoot{}

%%%%%%%%%%%%%%%%%%%%%%%%%%%%%%%%%%%%%%%%%%%%%%%%%%%%%%%%%%%%%%%%%%%%%%%%%%%%%%%%
\selectlanguage{portuges}

\Sumario
\ListadeSimbolos
\listoffigures
\listoftables
\lstlistoflistings %lista de codigos fonte - Para inserir a listagem de codigos fonte
\newpage
\cleardoublepage

\Introducao

%%%%%%%%%%%%%%%%%%%%%%%%%%%%%%%%%%%%%%%%%%%%%%%%%%%%%%%%%%%%%%%%%%%%%%%%%%%%%%%%
%
% Hifenizacao - Colocar lista de palavras que nao devem ser separadas e que 
% nao estao no dicionario portugues.
% As palavras do dicionario portugues ja sao separadas corretamente pelo lateX
%
\hyphenation{ Hardware Software etc  }


%%%%%%%%%%%%%%%%%%%%%%%%%%%%%%%%%%%%%%%%%%%%%%%%%%%%%%%%%%%%%%%%%%%%%%%%%%%%%%%%
%% A partir daqui coloque seus capitulos. Sugere-se que eles sejam inseridos com o comando \input
%% Da seguinte maneira:
%%
%% \input{cap1} 
%% \input{cap2}
\chapter{Introdu\c{c}\~{a}o}

\begin{itemize}
\item apresentação geral do assunto do trabalho;
\item definição sucinta e objetivo do tema abordado;
\item justificativa sobre a escolha do tema e métodos empregados;
\item delimitação precisa das fronteiras da pesquisa em relação ao campo e períodos abrangidos;
\item esclarecimentos sobre o ponto de vista sob o qual o assunto será tratado;
\item relacionamento do trabalho com outros da mesma área;
\item objetivos e finalidades da pesquisa, com especificação dos aspectos que serão ou não abordados;
\item a proposição poderá ser apresentada em capítulo à parte. 
\end{itemize}

\begin{itemize}
\item Vamos estudar aqueles usuários os quais seus colegas podem fácilmente inferir o gênero;
\item O que podemos aprender fazendo estas comparações?
\item Por que os posts das mulheres seria diferente dos feitos por homens?

\end{itemize}

Mulheres tendem a se identificar menos, mas isso não é relevante, já que nós queremos ver o impacto que "ser mulher" nestas comunidades tem na participação destes usuários.
%!TEX root = main_mestrado.tex
\chapter{Literatura}

Baseado na psicologia dos gêneros, encontramos duas teorias principais para justificar a ausência de mulheres em campos relacionados à tecnologia: a essencialista que vê as causas na biologia e o constructo social que vê as causas no meio, o qual é considerado "male-friendly" mas não "female-friendly". Contudo, ainda é importante levar em consideração as diferenças individuais destas pessoas ao estudar esse gender gap~\cite{trauth2004understanding}.

Nesta pesquisa estamos estudando gênero como fator social, como as pessoas se identificam, e não biológico, portanto usaremos o termo "gênero" ao invés de "sexo". Também estamos generalizando e estudando apenas os dois estremos do espectro: aqueles usuários que se identificam como homens e os que se identificam como mulheres. 

\section{Diferenças de gênero em comunidades online}

O comportamento de homens e mulheres em comunidades online nem sempre é o mesmo. Podemos observar diferenças até quanto ao uso da própria internet~\cite{hargittai2006differences} e a linguagem usada online. Esta última podendo influenciar na visão que cada gênero tem  de quão relevante e útil é o conteúdo da comunidade. Em outras palavras, a percepção de qualidade da comunidade~\cite{Gefen:2005:YSS:1066149.1066156}.

Em ambientes de aprendizado online, homens interagem mais com o ambiente e enviam mensagens mais longas, com um contexto mais social, do que as mulheres. Estas tendem a ter uma participação mais pontual e relacionada com o tema das aulas~\cite{barrett1999gender}. Já na Wikipedia, apesar de homens e mulheres terem comportamentos de contribuição similares, as mulheres são as que produzem contribuições mais longas~\cite{glott2010wikipedia}.

Mudando o foco para comunidades online mais relacionadas à tecnologia, foi encontrado que, em canais de IRC sobre diversos assuntos, usuários que apresentam um pseudônimo feminino tendem a receber 25 vezes mais mensagens maliciosas que usuários com pseudônimo masculinos e 4 vezes mais que usuários com pseudônimo que não se identificava o gênero~\cite{meyer2006assessing}. Isto pode ser um dos motivos pelo qual mulheres não participam tanto quanto homens deste tipo de comunidade. Contudo, observou-se também que comunidades que foram desenhadas para homens e que são conhecidas por não serem \textit{female-friendly} e serem cenário de inúmeros casos de perseguição às mulheres, como as comunidades MMORPG, registraram um crescimento na proporção de mulheres participantes, este número chegando até a ser maior que o de homens~\cite{taylor2003multiple}. 

Por fim, em comunidades \textit{Free Open Source} (FOSS) e na comunidade StackOverflow mulheres não são uma minoria, mas possuem uma baixa participação~\cite{rustad2011suck} e se engajam menos à comunidade~\cite{Vasilescu27092013}. Homens têm um tempo de vida maior nestas comunidades.

% TODO:
% Though we can't generalize women's behavior for every online community, even if they have similar shapes or similar purposes. And, as everyone of these communities can improve as their user diversity widen, it's important to study women's behavior in each one so we can make suggestions to solve its gender gap.

\section{Feedback e competitividade}

Feedback mostra-se como uma ferramenta que influi bastante no comportamento dos usuários em comunidades online. Sabe-se que feedback negativo pode trazer um efeito indesejado à comunidade, fazendo com os que o recebem mantenham-se ativos e produzindo conteúdo de baixa qualidade~\cite{cheng2014community}. O estudo mostra também que o efeito é viral: pessoas que recebem feedback negativo tendem a passá-lo para frente. Em contra partida, na Wikipedia, acredita-se feedbacks positivos são um fator importante na motivação de mulheres a participar da comunidade visto que este aumenta a auto-estima delas quanto ao seu conhecimento no assunto da contribuição~\cite{collier2012conflict}.

Com relação a competitividade, estudos em laboratório mostram que mulheres não são tão eficientes quanto homens em ambientes competitivos, principalmente quando é necessário competir com homens~\cite{gneezy2003performance}. As comunidades do StackExchange podem ser vistas como ambientes competitivos, visto que as respostas passam por um processo seletivo para cada pergunta. É possível que um baixo desempenho das mulheres nestas comunidades possa ser frustrante e levá-las a sair cedo da comunidade. Interessante notar que, enquanto a maioria das mulheres as evitam, homens sentem-se confortáveis em competições~\cite{niederle2005women,croson2009gender}.

\section{Mulheres, STEM e autoconfiança}
% autoconfiança = confidence.
Estudos mostram que mulheres têm performance tão boa quanto a de homens em atividades de matemática~\cite{hyde1990gender,campbell1986effects} e que até costumam ter notas melhores em disciplinas relacionadas à STEM~\cite{stoet2015sex}. Mais ainda, já se sabe que a performance das mulheres neste campo não depende de igualdade política, econômica ou social~\cite{stoet2015sex}. E ainda assim, o estereótipo de que mulheres são menos competentes em campos relacionados a STEM é bem visível e documentado~\cite{moss2012science}.

Apesar do seu potencial, mulheres ainda são minoria em campos relacionados a STEM que acredita-se que é necessário um talento inato para poder alcançar o sucesso na área~\cite{leslie2015expectations}. Este é o caso de Ciência da Computação e Filosofia, por exemplo. O motivo para que isto aconteça pode ser a crença por parte de ambos os gêneros que mulheres não possuem tal talento~\cite{tiedemann2000gender,kirkcaldy2007parental}. Apesar de que não foi provado ainda que esse "talento inato" é exclusivo de homens ou mulheres~\cite{hyde2005gender}. Contudo, se mulheres absorverem este estereótipo, elas podem achar que estes campos não são para elas~\cite{wigfield2000expectancy,shapiro2011major}.

O problema parece estar na falta de auto confiança por parte das mulheres que tendem a duvidar que possuem tal talento e sentem-se desencorajadas a tentar campos como matemática e filosofia~\cite{leslie2015expectations}. Já foi observado que mulheres realmente possuem baixa auto confiança em tarefas de ciências comparadas aos homens~\cite{fox1992confidence} e que resultados em tarefas de matemática afetam mais mulheres do que homens com relação a sua auto confiança em sua performance~\cite{campbell1986effects}. Esta falta de confiança em seu potencial já chegou até a afetar comunidades como a Wikipedia, onde a maior parte das contribuições vêm de uma minoria de homens~\cite{antin2011gender,lam2011wp}.

Por outro lado, estudos apontam que mulheres que têm contato com "role models" em ciência podem mudar a sua postura com relação à disciplina de uma maneira positiva~\cite{smith1986effect}, e isto pode ter um impacto positivo em seus interesses educacionais~\cite{nixon1999educational}.

% \section{Mulheres e carreiras em STEM}

Mulheres não consideram seu gênero tão influente quando decidem se querem seguir carreira em Ciência da Computação quanto condições culturais e o ambiente onde irão trabalhar~\cite{blum2007cultural}. Mas, uma baixa proporção de mulheres numa disciplina pode mandar uma mensagem negativa para mulheres, como se a disciplina não fosse interessante para as outras e que elas a deveriam evitá-la também~\cite{shapiro2011major}.

% Contudo, o preconceito com mulheres na área ainda é muito forte. Uma prova que tal estereótipo de que mulheres são menos competentes pode ser encontrado em ~\cite{moss2012science} onde acadêmicos de ambos os sexos não só preferem contratar homens à mulheres, como oferecem aos candidatos do sexo masculino um maior salário e melhor suporte de carreira (career mentoring). Isso se alinha com o estudo que mostra que homens consideram matemática como "coisa de homem" mais do que mulheres~\cite{hyde1990gender}.
% TODO: verificar em qual área é a pesquisa de moss2012science e colocar aqui.

% \bigskip

% TODO: Último parágrafo resumindo tudo. O antigo para me inspirar:
% Taken together, the literature points that although one should not expect women to have less ability in STEM-related tasks, the environment they live in and how they receive performance feedback seems to be determinant on how women relate to STEM. These same factors seem to play a role in some online communities. Our study contributes to this body of knowledge by investigating how women contribute in STEM Social Q\&A online communities. On the one hand, these are environments closely related to the STEM field. On the other hand, from the perspective of research, gender differences have been little explored in this point of the online communities design space. 


%!TEX root = main_mestrado.tex
\chapter{Metodologia}

Esta sessão detalha os dados e métodos utilizadas para medir quantidade, qualidade e frequência das contribuições dos usuários, assim como o compromisso dos usuários com a comunidade. Aqui também explicaremos como identificamos os gêneros dos usuários. Todo o código utilizado neste experimento pode ser encontrado em um repositório no Github\footnote{URL anonymized for submission}. 

\section{Dados utilizados}

A principio, nossa intenção era estudar todas as comunidades do StackExchange em seu último dump de Setembro de 2014. Contudo, foi detectado que algumas comunidades não possuíam atividades suficientes, por parte das mulheres, para serem analisadas. Portanto, estudamos todas as comunidades do StackExchange, presentes no dump de Setembro de 2014, que tenham 18 ou mais meses de idade, mais a comunidade "poker". A lista de comunidades estudadas pode ser encontrada aqui\footnote{colocar uma tabela? ou algo assim}. Mais ainda, apenas os usuários que tenham feito algum tipo de contribuição e têm, no mínimo, 50 pontos de reputação foram estudados. Esta limitação é para garantir que estudamos usuários que estiveram ativos em algum momento na comunidade estudada.

As informações utilizadas neste estudo foram obtidas no dump trimestral\footnote{https://archive.org/details/stackexchange} oferecido pelo StackExchange. O grupo costuma publicar regularmente todos os dados de todos os seus sites que não contenham informações privadas do usuário (e.g. e-mail). Tais dados incluem perguntas, respostas e comentários postados pelos usuários, assim como os votos (positivos e negativos) recebidos por cada tipo de post. O dump também contém a reputação de cada usuário em cada comunidade. 

Utilizamos os dados de todas as comunidades estudadas, desde o seu início até Setembro de 2014. Tais comunidades são dividas em seis categorias: Technology, Culture/Recreation, Life/Arts, Science, Business e Professional.
%TODO:A intenção é comparar o comportamento dos usuários das comunidades de Technology e Science com as demais.

\section{Inferindo o gênero dos usuários}

Os perfis do StackExchange não possuem explicitamente o gênero de cada usuário, já que a plataforma não pede para que o usuário forneça esta informação no seu perfil. Mais ainda, o site não exige que os usuários preencham seu nome verdadeiro. Sendo assim impossível identificar o gênero de cada usuário automaticamente, com precisão. Já que nosso objetivo neste estudo é apenas investigar o efeito de um usuário, explicitamente, manifestar seu gênero, decidimos focar apenas naqueles usuários que têm a intenção de identificar seu gênero nestas comunidades.

Claramente, este tipo de comunidade online possui três tipos de atividades principais: perguntar, responder e comentar. Durante a participação de um usuário nas atividades principais, só é possível identificar o gênero do seu colega por dois meios: sua foto ou nome de usuário. Utilizamos o nome de usuário para identificar o gênero do mesmo, visto que pesquisas anteriores mostram que, para pessoas com nomes ocidentais, inferir o gênero de uma pessoa através de sue primeiro nome é um método acurado tanto para redes sociais que requerem o nome real do usuário~\cite{tang2011s}, quanto para as que não o requerem~\cite{burger2011discriminating}\cite{liu2013s}.

Nosso método de inferência de gênero é similar ao encontrado ~\cite{liu2013s} e \cite{cunha2014he}. Nós utilizamos o Global Name Data~\cite{Hyland:2013:Online}, que é um banco de dados de nomes baseado em registros de nascimentos nos Estados Unidos e Reino Unido, e que contém mais de 100,000 nomes únicos. Com este banco de dados fizemos um classificador que categoriza os usuários em três classes baseado na contagem de frequência probabilística nos rótulos "Feminino", "Masculino" e "Desconhecido". Cada rótulo indica se a maioria das pessoas catalogadas com um certo nome é Homem ou Mulher. O gênero de um nome é considerado Desconhecido se não há diferença estatística entre a proporção de homens e mulheres catalogados com este nome. Nossa análise só leva em consideração aqueles usuários que possuem nomes os quais é possível inferir com confiança o gênero, de acordo com este método.

Com este método, conseguimos identificar, em média, 37\% dos usuários (mínimo: 27\%, máximo: 52\%) dos usuários das comunidades estudadas. Sendo sua grande maioria homens: em média, 93\% são homens (mínimo: 74\%, máximo: 98\%).

% TODO: Não sei se devo colocar o parágrafo abaixo. 
% It is worthwhile to note that users with western names are likely overrepresented in this sample. Nevertheless, given that a large proportion of StackExchange users are from western countries~\cite{schenk2013geo}, this sample is still likely representative of a large portion of the sites we study. 


\section{Variáveis}

Por motivos de espaço, um sumário de nossas métricas podem ser encontrado aqui\footnote{URL anonymized for submission}. A maioria delas possuem uma distribuição bastante enviesada e com certeza não normal. Portanto, utilizamos testes não paramétricos. Para todas as hipóteses da primeira e segunda pergunta, mais a segunda hipótese da terceira pergunta, foi utilizado o teste Mann-Whitney. Para testar se homens e mulheres possuem um tempo de vida divergente na comunidade utilizamos a \textit{Survival Analysis} junto com um teste de LogRank. Já para a quarta pergunta tomamos a proporção de contribuição e registros feitos por cada grupo de gênero, a cada seis meses, e verificamos se esta proporção aumenta ou não, com a ajuda de uma regressão utilizando o método dos mínimos quadrados.

\section{Medindo contribuições}

A questão principal do nosso estudo é esclarecer o quão diferente é a participação de homens e mulheres em comunidades de perguntas e respostas, comparando àquelas relacionadas a STEM (pertencentes às categorias Technology e Science) com as demais. Para responder esta questão, nós à dividimos em quatro perguntas mais específicas.

\subsection{Número de contribuições}

Nossa primeira pergunta é: \textit{O número de contribuições feitas por homens e por mulheres difere significantemente?} Esta pergunta é relacionada aos três tipos principais de contribuição na plataforma StackExchange: perguntas, respostas e comentários, além da soma de todas. As hipóteses testadas são:

\begin{itemize}
    \item \textit{H1$_0$: O número de perguntas dos usuários de ambos os gêneros pertencem à mesma distribuição.}
    \item \textit{H2$_0$: O número de respostas dos usuários de ambos os gêneros pertencem à mesma distribuição.}
    \item \textit{H3$_0$: O número de comentários dos usuários de ambos os gêneros pertencem à mesma distribuição.}
    \item \textit{H4$_0$: O número de contribuições dos usuários de ambos os gêneros pertencem à mesma distribuição.}
\end{itemize}

\subsection{Qualidade}

O controle de qualidade das perguntas e respostas na plataforma StackExchange é feito pela própria comunidade através do sistema de votos. Usuários com uma determinada reputação podem votar "para cima" e "para baixo", caso considerem o post como uma boa ou má contribuição, respectivamente. Esses votos geram pontos, positivos ou negativos, que, dentre outros fatores, formam a reputação de um usuário. 

Tendo em vista o parágrafo acima, a nossa segunda pergunta é: \textit{As contribuições feitas por diferentes gêneros são vistas, pela comunidade, com níveis de qualidade diferentes?} Esta pergunta se refere ao fato de que mulheres podem estar contribuindo menos porque suas contribuições não são bem vindas, fazendo com que elas sintam que falharam e as desencorajando de continuar participando. 

Utilizamos a pontuação criada pelo sistema de votos para medir a qualidade das perguntas e respostas de cada usuário. A qualidade das perguntas de um usuário é definida como a média da pontuação de todas as perguntas feitas pelo usuário, ou zero, caso o usuário tenha postado nenhuma pergunta. Para respostas, utilizamos duas métricas: a primeira é o \emph{Accepted Rate} que mede a proporção de respostas do usuário que foram aceitas como melhor resposta pelo interrogador. A segunda é a \emph{Mean Utility of Answers}, proposta por Furtado et al.~\cite{furtado2013contributor}. Esta métrica é "averages the standardized score of votes answers by a user received compared to competing answers in the same questions". Para ambas as métricas para respostas, apenas usuários com pelo menos uma resposta são levados em consideração.

% TODO: traduzir o significado de mean utility.

As hipóteses que consideram a qualidade do conteúdo do usuário são as seguintes:

\begin{itemize}
    \item \textit{H1$_0$: A média da pontuação das perguntas de cada usuário, de ambos os gêneros, segue a mesma distribuição.}
    \item \textit{H2$_0$: O accepted rate dos usuários de ambos os gêneros pertencem à mesma distribuição.}
    \item \textit{H3$_0$: A mean utility dos usuários de ambos os gêneros pertencem à mesma distribuição.}
\end{itemize}

\subsection{Engajamento}

Neste estudo, usamos a definição de engajamento a qual o descreve como o esforço de um usuário ao longo do tempo. Nós medimos engajamento tanto em relação ao tempo total que o usuário permanece na comunidade (seu tempo de vida) quanto sua frequência de contribuição durante seu tempo de atividade.

Levando em consideração estas definições, nossa terceira pergunta é: \textit{O engajamento de homens e mulheres difere nestas comunidades?} É importante notar que caso haja diferença no hábito de contribuição entre homens e mulheres, podemos verificar se usuários de algum dos gêneros deixam a comunidade mais cedo ou têm uma taxa de contribuição menor, nos ajudando a entender melhor as diferenças (ou similaridades) entre homens e mulheres com relação ao seus hábitos de contribuição.

O tempo de vida de um usuário em uma comunidade do StackExchange não pode ser considerada apenas como a diferença entre a data de sua primeira contribuição e a última. Já que uma pergunta (e suas respectivas respostas) podem ser transferidas entre comunidades\footnote{http://meta.stackexchange.com/questions/2683/move-questions-between-stack-exchange-sites}, nós definimos o tempo de vida de um usuário em uma determinada comunidade como segue: antes de tudo, definimos o primeiro dia de atividade do usuário como a data mais recente entre o dia de cadastro do usuário e a data da sua primeira contribuição na comunidade. Esta medida evita que levemos em consideração um período que o usuário não esteve ativo na comunidade em questão, mas sua pergunta foi transferida de outra comunidade. Mais ainda, definimos que um usuário está inativo quando ele não produz nenhuma contribuição por um período de tempo maior que o intervalo de morte da comunidade. Este intervalo, medido em dias, é calculado da seguinte maneira: para cada usuário obtemos o maior intervalo entre duas de suas contribuições consecutivas. O intervalo de morte é a média dos maiores intervalos de todos os usuários. Por fim, o tempo de vida de um usuário é a diferença entre a data da sua inatividade e a data da sua primeira atividade(ou cadastro), medida em dias.

A frequência de participação, segunda métrica usada nesta pergunta, é definida como o total de contribuições feita por um usuário (perguntas, respostas ou comentários) dividido pelo número de dias nos quais o usuário estava ativo. Ao contrário do tempo de vida, aqui nós consideramos apenas os dias em que o usuário fez algum tipo de contribuição. Para esta pergunta, as hipóteses são:

\begin{itemize}
    \item \textit{H1$_0$: O tempo de vida do usuário é o mesmo entre os gêneros.}
    \item \textit{H2$_0$: A frequência de participação dos usuários de ambos os gêneros pertencem à mesma distribuição.}
\end{itemize}

\subsection{Contribuições e registros ao longo do tempo}

Para esta pergunta, queremos verificar se a falta de contribuição por parte das mulheres é algo novo nestas comunidades, ou ainda se existem padrões de contribuição ao longo do tempo, seja de aumento ou diminuição. Em outras palavras, nós queremos responder à seguinte pergunta: \textit{A proporção de contribuições feitas por usuários de cada gênero tem aumentado?}. Nós também investigamos se existe alguma tendência no número de registros feito por cada gênero, respondendo à seguinte pergunta: \textit{A proporção de registros feitos por usuários de cada gênero tem aumentado?}

\begin{itemize}
    \item \textit{H1$_0$: A proporção de contribuições feitas por homens e mulheres não muda ao longo do tempo.}
    \item \textit{H2$_0$: A proporção de registros feitos por homens e mulheres não muda ao longo do tempo.}
\end{itemize}

% section methods (end)



% \begin{itemize}
% 	\item O material utilizado para esta pesquisa está no github.
% 	\item Nesta sessão irão as hipóteses e questões de pesquisa.
% \end{itemize}

% \section{Dados Utilizados} % (fold)
% \label{sub:dados_utilizados}
% Os dados utilizados 
% \begin{itemize}
% 	\item Foi utilizado o dump disponibilizado pelo StackExchange no Archive.org datado de Setembro de 2014. Falar sobre licensa.
% 	\item Todas as comunidades foram estudadas. As comunidades são divididas em cinco categorias: Tecnologia, Cultura e Recreação, Negócios, Vida e Artes, Profissional e Ciência.

% \end{itemize}

% % section dados_utilizados (end)

% \section{Classificador de Gênero} % (fold)
% \label{sub:classificador_de_g_nero}
% \begin{itemize}
% 	\item Colocar exemplos de inferência de gênero utilizando o classificador automático;
% 	\item Não há identificação de gênero no perfil do usuário no StackExchange, então o gênero precisa ser inferido;
% 	\item São estudados aqui apenas os usuários ativos e aqueles que podemos facilmente identificar seu gênero;
% 	\item As principais atividades de um usuário nestes sites são: comentar, perguntar e responder;
% 	\item Durante a participação de um usuário nas atividades principais, só é possível identificar o gênero do seu colega por dois meios: foto ou nome;
% 	\item resolvemos usar seu nome para identificar o gênero dos usuários.
% 	\item Previous research shows that for people with western names, assessing sex based on the first name of an individual is accurate in social networks -- both those which require real names~\cite{tang2011s} and those which don't~\cite{burger2011discriminating}\cite{liu2013s}. 
% 	\item Nosso método de inferência é similar ao utilizado em \cite{cunha2014he}.
% 	\item We used a classifier which categorizes users into 3 classes based on the probabilistic frequency count under labels Male, Female, Unknown; each label indicates if the majority of people cataloged with a certain name is Male or Female. A name is considered Unknown if there’s no statistical difference between the proportion of Male and Female people cataloged with a certain name.
% 	\item descrever como funciona o método de identificação de gênero
% 	\item falar sobre o Global Name Data
% 	\item justificar porque essa amostra é representativa
% 	\item tabela ou texto sobre proporção identificada nas comunidades.
% \end{itemize}

% % section classificador_de_g_nero (end)

% \section{Contribuições em Número} % (fold)
% \label{sub:contribui_es_em_numero}

% The first question is: \textit{Does the amount of contribution from the two genders significantly differ?}
% This question is addresses the three central types of contribution in the StackExchange platform: questions, answers and comments. And, also, the sum of them. The hypotheses tested are:
    
% \begin{itemize}
%     \item \textit{H1$_0$: The median number of questions a user posts is the same for the two genders.}
%     \item \textit{H2$_0$: The median number of answers a user posts is the same for the two genders.}
%     \item \textit{H3$_0$: The median number of comments a user posts is the same for the two genders.}
%     \item \textit{H3$_0$: The median number of total contributions from a user is the same for the two genders.}
% \end{itemize}

% % section contribui_es_em_numero (end)

% \section{Qualidade} % (fold)
% \label{sub:qualidade}

% \textbf{Obs.:} As comunidades ham, startups e poker foram removidas da Q2 por não ter atividade suficiente por parte das mulheres (0). São comunidades atípicas. Nenhuma mulher fez pergunta, ou só uma mulher postou resposta. Ham e startups são comunidades novas.

% Our second question is: \textit{Are contributions from different genders perceived to have different quality levels by the community?}. This question deals with the fact that women may contribute less because their contributions are less appreciated and that may be felt as a failure, discouraging further contributions. 
% % TODO: essa frase merece uma reescrita
% Each content created in StackExchange possesses a measure of its quality as perceived by the community created by the collective feedback the community gives through up and down votes. 

% We use the score created through this voting to measure the quality of each user's answers and questions. The voting score of comments was not considered, as their quality is less objective and instrumental in the site. 

% A user's quality in creating questions is defined as the mean of all scores of all questions the user created, or zero in case the user has created no questions. For answers, two metrics are used: first, the \emph{Accepted Rate}, which measures the proportion of answers of users that were chosen as the best answer by the questioner. The second metric is the  \emph{Mean Utility of Answers}, as proposed by Furtado et al.~\cite{furtado2013contributor}, which averages the standardized score of votes answers by a user received compared to competing answers in the same questions. For both answer quality metrics, only users that provided at least one answer are considered.

% The hypotheses considering quality of contributions are thus:

% \textit{H1$_0$: The mean of the scores of the questions made by each user doesn't change between genders.}

% \textit{H2$_0$: Both genders have the same answer acceptance rate.}

% \textit{H3$_0$: The mean utility of the answers from a user is the same between genders.}

% % section qualidade (end)

% \section{Dedicação} % (fold)
% \label{sub:dedica_o}

% % Parágrafo abaixo pode ser melhor escrito
% Among the various meanings that the expression "engagement" can relate to, in this study we use it to define the effort put by a user in the system over time. We measure engagement both in terms of the total timespan during which the user was contributing to a site and as the frequency of contributions during this period. 

% Using these definitions, our final question is: \textit{Is the engagement of men and women, with those communities significantly different?} 
% % Mais coisa que pode ser reescrita porque faz sentido algum
% Note that this step is necessary to further understand differences or similarities in the contribution behavior of women and men. Otherwise, if there is a difference in total contribution between the genders, it would not be possible to know whether this happens because men/women leave earlier or have lower contribution rates. Additionally, understanding whether women are led to leave the community sooner than men can shed light on exclusion phenomena. 

% The lifetime of a user is the difference from the first day of activity and the last. As a question (followed by it's answers) can be transferred between communities\footnote{http://meta.stackexchange.com/questions/2683/move-questions-between-stack-exchange-sites} we define the first day of activity as the most recent date between the user's join date and the first contribution's date on the community in question. This avoids taking into account the time between the original transferred question's date and the user's join date, as this time the user can't be active in the community. We also define that a user has become inactive in a community after the user is not active for a period larger than the death interval for the community. This interval, measured in days, is calculated as the following: for each user, we take the largest interval between two contributions by this user. The death interval is the average of this measure for all users.

% The frequency of participation is the second metric used in this question, and it is defined as the amount of contributions made by a user (question, answer or comment) divided by the amount of days in which the user was active. Opposed to our lifetime metric, here we only take into account the days in which the user made any kind of contribution. Our resulting hypotheses are:

% \textit{H1$_0$: The lifetime of users is the same between genders.}

% \textit{H2$_0$: The frequency of participation is the same between genders.}

% % section dedica_o (end)

% \section{Contribuições e Cadastros ao Longo do Tempo} % (fold)
% \label{sub:contribui_es_e_cadastros_ao_longo_do_tempo}

% \begin{itemize}
% 	\item Comunidades com menos de 14 meses foram retiradas porque a regressão estava dando errado (NaN). Acredito que por que foram separados em semestres e 2 semestres (ou 2 pontos) era muito pouco para a regressão
% 	\item Proporção de contribuições ao longo do tempo, não cumulativo e proporção de registros ao longo do tempo, não cumulativos.
% \end{itemize}

% For this research question we want to verify if the lack of female contribution is something new on these communities. Or even if there are any patterns of rise or fall of contributions over time. In other words, we want to answer the following question: \textit{Do the proportion of contributions by each gender change over time?} Also, we investigated if there are any trends on registrations by each gender, answering the following question: \textit{Do the proportion of registrations by each gender change over time?}

% \textit{H1$_0$: The proportion of contributions by men and women hasn't changed over time.}

% \textit{H2$_0$: The proportion of registrations by men and women hasn't changed over time.}
% % section contribui_es_e_cadastros_ao_longo_do_tempo (end)

% \section{Variable distributions}
% \label{sub:distributions}
% \begin{itemize}
% 	\item Simples justificativa por que cada teste é usado em cada hiposese
% 	\item Aqui havia uma tabela com o sumarize de cada métrica em cada comunidade. Acho que não é viável fazer isso aqui, mas é importante citar, para poder  justificar mann whitneey e ouutros testes.
% 	\item summary das variáveis
% 	\item Falar sobre as regressões e survival analysis. 
% \end{itemize}
% % Our metrics are summarized in Table \ref{tab:summary}. Most of our metrics have highly skewed distributions. The observation that these are non-normal distributions leads to the use of Mann-Whitney two sample tests for answering all hypotheses but the one regarding lifetime. For testing if men and women have different lifetime lengths, a survival analysis along with a LogRank test were employed.
%!TEX root = main_mestrado.tex
\chapter{Resultados}

\begin{itemize}
	\item 
\end{itemize}

\section{Número} % (fold)
\label{sec:numero}

\begin{itemize}
	\item Q1H1: Em 25 de 29 comunidades mulheres perguntam mais do que homens. Excessões: homebrew, german, poker, gardening. Maioria em que homens perguntam mais é da categoria culture-recreation. Apenas uma comunidade onde mulheres perguntam mais é de culture-recreation.
	\item Q1H2: Mulheres 20/34, 14/34 homens respondem mais. Das comunidades que homens respondem mais, a maioria está nas categorias technology, science. Das comunidades que mulheres responderam mais, a maioria está em culture-recreation e life-arts. As comunidades de tecnologia e ciência que as mulheres respondem mais são conhecidas por ter mais mulheres na área (ux, sqa, biology)
	\item Q1H3: Homens comentam mais em 8/17 e mulheres em 9/17. Maioria das mulheres em technology e maioria dos homens em culture-recreation.
	\item Q1H4: Mulheres tem maiores contribuições em 14/19 e homens em 5/19. Homens tem maioria de contribuições em comunidades culture-recreation e science. Mulheres em technology.
\end{itemize}

% section n_mero_de_contribui_es (end)

\section{Qualidade} % (fold)
\label{sec:qualidade}

\begin{itemize}
	\item Q2H1: Mulheres têm accepted rate maior em 8/13 e homens em 5/13.  Não existe uma maioria.
	\item Q2H2: Mulheres têm mean utility maior em 10/13 comunidades e homens em 2/13 (stackoverflow e dba). ham deu NaN.
	\item Q2H3: Mulheres têm avg score de questions maior em 6/18 comunidades e homens em 10/18. startups e poker deram NaN. Mulheres: culture-recreation e technology. Homens technology. Essa foi a única hipótese que as medianas não bateram com o p-valor.

As comunidades que deram NaN foi por que não havia atividades de mulheres naquele determinado critério.

\end{itemize}

% section qualidade (end)

\section{Dedicação}% (fold)
\label{sec:dedicacao}

\begin{itemize}
	\item Q3H1: Lifetime maior para homens em todas as comunidades menos wordpress. 18/19 e 1/19
	\item Q3H2: Apenas duas comunidades homens têm frequencia de postagem maior do que mulheres: chess e movies. 12/14 e 2/14
\end{itemize}

Lembrar de citar ~\cite{Vasilescu27092013} aqui sobre o tempo de vida no SO.

% section dedicacao (end)

\section{Contribuições Ao Longo do Tempo}% (fold)
\label{sec:tempo}

\begin{itemize}
	\item Q4H1: Na maioria das comunidades a contribuição feita por mulheres está aumentando (20/27 - 7/27) a maioria na categoria technology e culture-recreation. Homens contribuem mais em technology tbm. Mas, coeficientes são ínfimos. 
\end{itemize}

% section tempo (end)

\section{Registros Ao Longo do Tempo}% (fold)
\label{sec:registros}

\begin{itemize}
	\item Q5H1: Comunidades que a quantidade de registros de homens têm aumentado: 4/12 patents, blender, chinese, poker. Maioria que mulheres estão se cadastrando mais é na categoria technology. Anyway, coeficientes ainda são bem pequenos.
\end{itemize}

% section registros (end)

%!TEX root = main_mestrado.tex
\chapter{Discussão}
\label{ch:discussao}

Neste capítulo iremos comentar os resultados obtidos no nosso experimento e associá-los à literatura relacionada ao estudo aqui descrito. Dividimos nossos pensamentos entre as semelhanças e diferenças encontradas e, apesar da importância do primeiro tópico, focamos nas diferenças por estas serem mais controversas.

\section{As semelhanças}

Nossa expectativa ao observar contribuições individuais de usuários de diferentes gêneros era que mulheres, por serem uma minoria, iriam participar menos destes sites, assim como na \emph{Wikipedia}~\cite{antin2011gender} e comunidades \emph{FOSS}~\cite{rustad2011suck}. Contudo, o que encontramos não foram diferenças, mas uma grande equivalência entre o comportamento de homens e mulheres nos sites estudados. 

No geral, saber que mulheres contribuem de maneira similar aos homens na maioria dos sites aponta que uma mulher que se cadastrar em um site aleatório do \emph{StackExchange} provavelmente irá encontrar um ambiente que não a estimule a contribuir menos ou menos frequentemente que os homens. Estes resultados aparentam inesperados visto a pequena proporção de mulheres na maioria dos sites.

\section{As diferenças}

Tendo em mente todo o estereótipo em \emph{STEM} que mulheres são enquadradas, esperávamos que sua baixa representatividade também representaria baixa contribuição. Ademais, esperávamos que o estereótipo de mulheres em tecnologia~\cite{hyde1990gender} e aquele em que mulheres não possuem um taleto inato para ciência~\cite{leslie2015expectations} fosse influenciar na opinião que os outros usuários teriam das suas colegas, fazendo com que mulheres recebessem \emph{feedback} negativo nas suas contribuições, indiscriminadamente, levando o conteúdo das mulheres a serem classificados com menor qualidade pela comunidade. Sem esquecer de que outros estudos já tinham identificado o conteúdo vindo de mulheres como de qualidade inferior comparado ao de homens, tanto em tarefas de matemática~\cite{campbell1986effects} quanto em comunidades online~\cite{collier2012conflict}.

Contudo, não é isso que acontece. Na maioria dos sites, mulheres produzem conteúdo de tão boa qualidade quanto homens, segundo a visão da comunidade. Vemos que estas mulheres contribuem bastante com perguntas e comentários, tendo uma frequência de contribuições muitas vezes maiores que a de homens. 

% Por outro lado, apesar de mulheres respondem mais do que homens em várias comunidades, quando observamos apenas comunidades relacionadas a \emph{STEM} esta proporção é invertida. Inclusive, mulheres respondem menos nos quatro maiores e mais antigos sites da plataforma: StackOverflow, SuperUser, SeverFault e Mathematics. Mais ainda, em praticamente todas as comunidades que observamos diferença estatística quanto ao tempo de vida, homens passavam mais tempo na comunidade.

% Quanto ao tempo de vida de mulheres nestas comunidades, nota-se que a diferença de tempo é maior em comunidades mais antigas, vide tabela \ref{table:lifetime}. Isso pode se dar por que tais comunidades, a exemplo do StackOverflow, foram criadas por um grupo de homens e se mantiveram assim por bastante tempo, até se tornarem popular. Este grupo de usuários provavelmente tem um tempo de vida bem maior do que as mulheres que eventualmente ingressaram na comunidade. 

Por outro lado, apesar de mulheres respondem mais do que homens em várias comunidades, quando observamos apenas comunidades relacionadas a \emph{STEM} isto não acontece. Mulheres contribuem com menos respostas do que homens em todos os quatro maiores e mais antigos sites da plataforma: \emph{StackOverflow}, \emph{SuperUser}, \emph{SeverFault} e \emph{Mathematics}. Se estes sites influenciam mulheres a participarem menos no processo de resposta, a grandeza deles pode implicar que um grande número de mulheres estão contribuindo menos respostas do que poderiam estar. Sabendo que existem sites em outras categorias onde mulheres comuns respondem mais do que homens, nossos resultados sugerem que há um fator característico em alguns sites -- e provavelmente à comunidade -- atuando no comportamento das mulheres quanto às suas contribuições. A visível e gritante diferença no tempo de vida entre homens e mulheres nestas comunidades complementa este pensamento (ver Tabela~\ref{table:lifetime}).

Uma característica comum a algumas comunidades relacionadas a \emph{STEM} do \emph{StackExchange} é que elas são ramos do \emph{StackOverflow} que foi criada por um grupo de homens e se manteve assim por bastante tempo, até se tornar popular. É provável que este grupo de homens tenham sua própria cultura dentro do \emph{StackOverflow}, com seus valores e práticas que não sejam amigáveis às mulheres, e esta cultura tenha passado para os demais sites, fazendo com que estes sejam sites com características sociais que não atraiam as mulheres.

Nosso resultado se alinha com o de Yang \textit{et al}.~\cite{yang2010activity}, que apesar de não estudar a mesma comunidade de perguntas e respostas, defendeu que usuários que têm o tempo de vida maior na comunidade preferem responder à perguntar. No nosso estudo, verificamos que estes usuários são geralmente homens, assim como entendeu Vasilescu \textit{et al}.~\cite{Vasilescu27092013}. Vasilescu \textit{et al} considera este resultado insalubre para a comunidade, visto que bons contribuidores estão deixando a comunidade. Nós acreditamos que exista uma barreira para que mulheres continue contribuindo, podendo ser desde a aquisição, por parte delas, do pensamento de que não são capazes de contribuir, até por perseguições/assédios dentro da própria comunidade por parte dos outros usuários. Estas hipóteses, contudo, pedem uma investigação mais detalhada, fora do escopo desta pesquisa.

Apesar de não ser relacionada com sua real performance, a percepção que uma pessoa tem de sua capacidade cognitiva em um campo influencia como esta pessoa acha que se sairá em uma determinada tarefa. No mais, achar que não vai se dar bem em uma tarefa pode fazer com que, caso seja dada a opção, a pessoa não chegue a tentar realizar tal atividade. Mulheres acham que não são boas em tarefas sobre ciência e isso as influencia a não participar do campo~\cite{ehrlinger2003chronic}. Isso pode explicar porque a proporção de mulheres é tão baixa nas comunidades sobre \emph{STEM}, mas elas se dão tão bem quanto homens na maioria delas.

% Uma justificativa para mulheres responderem menos em comunidades de \emph{STEM} pode ser a de que respostas podem ser consideradas mais competitivas do que perguntas. Afinal, todas as respostas passam por um "processo seletivo" para ser a melhor resposta de uma determinada pergunta, o que não acontece com perguntas nem comentários. Já se sabe que mulheres não se dão bem em ambientes competitivos, principalmente quando precisam competir com homens~\cite{gneezy2003performance}, e tendem a evitá-los~\cite{niederle2005women,croson2009gender}. Além de que, demonstrar conhecimento em um assunto desta área pode exigir uma auto-confiança no campo que mulheres não têm.

Se olharmos para comunidades de áreas de \emph{STEM} que são conhecidas por terem mais mulheres, observamos que o resultado é similar aos de comunidades não relacionadas à \emph{STEM}. Tomando como exemplo a comunidade \emph{UX} (\textit{user experience}) cujas variáveis podem ser observadas na Figura~\ref{fig:ux}, mulheres respondem mais nesta comunidade, o que pode ser uma demonstração de auto confiança no seu conhecimento no assunto. Mais ainda, \emph{UX} não é considerada uma disciplina que necessita de um conhecimento inato para ter sucesso na área, sendo este mais um motivo para mulheres não se inibirem em participar desta comunidade de todas as formas possíveis. 

\begin{figure}
  \centering
  \includegraphics[height=0.65\paperheight]{figures/ux.pdf}
  \caption[Distribuição das variáveis para o site \emph{User Experience}]{Distribuição das variáveis para o site \emph{User Experience}. Mostramos um histograma do log do valor da variável mais um para as variáveis discretas e a gráfico da função de densidade, para as variáveis contínuas. }~\label{fig:ux}
\end{figure}

Ver que mulheres têm aumentado sua proporção de contribuições ao longo do tempo, principalmente em comunidades sobre assuntos relacionados à \emph{STEM}, nos dá uma esperança que a barreira que existe para que mulheres entrem e contribuam para estas comunidades esteja diminuindo. Já que outras comunidades, como as comunidades \emph{FOSS}~\cite{powell2010gender}, mulheres que já estão envolvidas tendem a continuar fazendo parte dela. Uma maneira de aumentar a autoconfiança destas mulheres quanto ao seu conhecimento também é essencial para aumentar estas proporções, já que elas mesmas consideram ter autoconfiança algo essencial para se manter em comunidades que elas são maioria~\cite{powell2010gender}.

No geral, nossos resultados mostram que a maioria dos sites do \emph{StackExchange} possuem comunidades com poucas diferenças de comportamento relacionadas ao gênero. Porém, conseguimos enxergar claramente duas barreiras para a participação de mulheres nestes sites: uma na entrada, considerando a baixa proporção de contribuidoras e outra relacionada à partida antecipada de mulheres engajadas. Ambas as barreiras são mais expressivas nas maiores comunidades relacionadas a \emph{STEM}, sugerindo que nestas comunidades, mulheres sentem ainda mais dificuldade em participar. Todos esses pensamentos, juntamente com o fato de que mulheres tendem a contribuir com mais perguntas do que homens nestas comunidades, requerem uma investigação qualitativa aprofundada de modo a entender por que este padrão existe.


% Esse parágrafo é interessante, mas não acho que cabe aqui.
% However, in Wikipedia, it has been found that women are more likely to be involved in social- and community-oriented areas of the site~\cite{lam2011wp}. This is in line with women being more likely to comment in the communities we study, where comments are the primarily means of socialization. Nevertheless, the higher likelihood of women to produce questions in three of the communities remains an open question for future work. A possible explanation to be considered is that self-selection leads more men contribute as answering-only users in these communities.

% \subsection{Discussão do design}

% O que nosso estudo deixa bem claro é que as comunidades do \emph{StackExchange} possuem duas barreiras para a contribuição por parte das mulheres: um para que elas entrem e comecem a participar da comunidade e outra para que elas permaneçam tanto tempo quanto seus colegas do sexo masculino. As comunidades do \emph{StackExchange} pode estar perdendo muito, em termos de conteúdo, com essa postura por parte das mulheres em seu sites.

% É provável que exista preconceito e perseguição dentro destas comunidades sim, assim como boa parte das comunidades onlines. Identificar e eliminar usuários que não têm um comportamento de acordo com os termos da comunidade é uma boa prática. Contudo, é necessário mostrar para as mulheres que isto está sendo feito, para que elas possam se sentir seguras neste ambiente. 

% Mais ainda, uma campanha feminista (lembrando que o feminismo promove a igualdade entre os sexos e não a dominação do sexo feminino) para promover o site pode ser um bom primeiro contato com a comunidade feminina e procurar saber diretamente delas como melhorar.

% subsection sugestoes (end)


% Note to self: estou começando a achar que existe prejudice e harassment sim, mas as mulheres que conseguem se engajar com estas comunidades de tecnologia conseguem ignorar essas negatividades. Newcomers podem ver isto como barreira e nem se quer entrar na comunidade.

% "StackOverflow was founded by men, and this community may continued as a group of men until reach a certain popularity, number of users and before the birth of \emph{StackExchange} along with other communities, as most of them come from branches of StackOverflow. "

% "This pattern of data suggests, by extension, that women might disproportionately avoid scientific pursuits because their self-views lead them to mischaracterize how well they are objectively doing on any given scientific task. Because they think they are doing more poorly than do men, they are more likely than men to avoid science when given an option." Texto de ehrlinger2003chronic
\chapter{Conclusões}

\begin{itemize}
\item 
\end{itemize}
% \input{cap1}


%%%%%%%%%%%%%%%%%%%%%%%%%%%%%%%%%%%%%%%%%%%%%%%%%%%%%%%%%%%%%%%%%%%%%%%%%%%%%%%%
%% Bibliografia
%% Coloque suas referencias no arquivo ref.bib e descomente as proximas duas linhas

\bibliographystyle{plain} % estilo de bibliografia   plain,unsrt,alpha,abbrv.
\bibliography{ref} % arquivos com as entradas bib.

%%%%%%%%%%%%%%%%%%%%%%%%%%%%%%%%%%%%%%%%%%%%%%%%%%%%%%%%%%%%%%%%%%%%%%%%%%%%%%%%
%% Apendice
% Caso seja necessario algum apendice, descomente a proxima linha.

\appendix
\chapter{Meu primeiro apêndice}
\label{app:primeiro}

\chapter{Meu segundo apêndice}
\label{app:segundo}

%%%%%%%%%%%%%%%%%%%%%%%%%%%%%%%%%%%%%%%%%%%%%%%%%%%%%%%%%%%%%%%%%%%%%%%%%%%%%%%%

\end{document}