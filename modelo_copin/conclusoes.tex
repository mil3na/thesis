%!TEX root = main_mestrado.tex



\chapter{Conclusões}
\label{ch:concl}

Enquanto estudos de gênero em comunidades de perguntas e respostas costumam focar em número de usuários de cada gênero, esta pesquisa tomou uma abordagem diferente, estudando a participação, qualidade das contribuições e engajamento de usuários de ambos os gêneros. No geral, nossos resultados nos mostram que, individualmente, mulheres que identificam seu gênero, se comprometem aos sites tanto quanto homens. Em vários sites, mulheres contribuem até mais do que homens.

Semelhante à \emph{Wikipedia}~\cite{lam2011wp}, mulheres contribuem com conteúdo relevante para sites de perguntas e respostas, mas a grande maioria das contribuições ainda vêm de homens. Nosso estudo nos revela que, depois de passar um tempo contribuindo para os sites que estudamos, mulheres parecem encontrar uma barreira para continuar produzindo conteúdo de valia e continuar engajadas tanto quanto os homens. Não é objetivo do nosso estudo investigar porque há tão poucas mulheres dentro destas comunidades, mas saber que ambientes diversos beneficiam o funcionamento e qualidade de uma comunidade, nos leva a concluir que não incluir ou não motivar mulheres a entrar nestas comunidades pode ser um grande desperdício de contribuidores que podem manter, ou até aumentar, a qualidade dessas comunidades.

Um aspecto, que gostaríamos de destacar, destes resultados é a percepção de quão ativas mulheres podem ser nestas comunidades. Este fato pode ser aproveitado para incentivar a participação de mulheres nestes sites. Apresentar as contribuidoras que encontramos nestas comunidades como modelo de comportamento para outras mulheres pode ser de grande ajuda para aumentar a diversidade de gênero nestas comunidades, visto que mulheres tendem a ter uma boa resposta a este tipo de modelo~\cite{smith1986effect,nixon1999educational}. Mais ainda, saber que mulheres conseguem progredir e ter sucesso nestas comunidades pode fazer com que mulheres que apenas observam os sites se identifiquem com as outras e mudem sua percepção de o quão bem podem se adaptar, fazendo com que mudem sua opinião com relação a participar da comunidade~\cite{ehrlinger2003chronic}.

Em resumo, nosso estudo contradiz várias crenças comuns, além algumas pesquisas anteriores relacionadas a gêneros, \emph{STEM} e sites de perguntas e respostas. Isso nos soa como um reforço pertinente que o problema da disparidade de gênero pode aparecer de várias formas e não existe uma solução única para todas as versões do problema. Em nosso caso, observamos que mulheres precisam enfrentar uma barreira para começar a contribuir ou para se identificar como mulher nestas comunidades. Às vezes, uma barreira até em continuar contribuindo uma vez engajada ao site. Contudo, menos esforço parece ser necessário para manter a contribuição proveniente de mulheres já engajadas.

Por fim, é importante notar que o conhecimento preciso das diferenças comportamentais entre homens e mulheres com relação às contribuições para sites de perguntas e respostas não é só relevante para reduzir a disparidade na proporção de mulheres nestas comunidades. Esperamos que este estudo também sirva como arcabouço para iniciar e melhorar estudos de diferenças entre outras minorias e assim aumentar a diversidade de forma geral em comunidades de perguntas e respostas.

\section{Limitações}
\label{ch:limites}
Apesar de todas as resoluções as quais chegamos nesta pesquisa, ela não é completa. Uma pesquisa qualitativa soa como um bom começo para complementar as análises que começamos aqui. Outrossim, adicionar mais fatores relacionados à participação pode ajudar a ampliar a visão que temos da situação.

Quanto à inferência de gênero, limitamos nossa amostra àqueles usuários que identificaram a si mesmo com nomes contidos no \emph{Global Name Data}. Esta abordagem pode ter descartado usuários de outros países cujos nomes não são comuns nos registros de nascimento dos Estados Unidos e Reino Unido. Uma pesquisa futura poderia realizar as mesmas comparações que fizemos, mas utilizando um outro método de identificação de gênero, como a foto do perfil do usuário.

Embora nossa amostra contenha sites com uma diversidade significativamente de temas, todos eles pertencem a uma mesma plataforma: o \emph{StackExchange}. Comparar nossos resultados com outros sites que tenham um propósito similar, mas um design diferente pode ajudar a entender melhor a relação entre o design do site, a diversidade de gêneros da comunidade e o comportamento de seus usuários.

Finalmente, nossa abordagem quantitativa e conjunto de dados usado não nos permite investigar profundamente quanto às razões de várias diferenças e similaridades encontradas. Acreditamos ser necessário e oportuno realizar um estudo qualitativo que complementasse nossa iniciativa.