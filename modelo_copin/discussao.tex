%!TEX root = main_mestrado.tex
\chapter{Discussão}

Bem como as diferenças, é importante observar também as não diferenças. Saber que, na maiorias das comunidades e na maioria dos fatores estudados, não encontramos diferença estatisticamente relevante entre homens e mulheres nos leva a crer que as comunidades do StackExchange não são tão hostil com as mulheres quanto se espera ao ver a proporção delas nos sites. 

A discussão aqui neste artigo, todavia foca nas diferenças e nas comunidades onde vemos diferenças, afinal é nelas que podemos ver os pontos fracos desses sites, descobrir problemas e sugerir soluções.

\subsection{Science, Technology, Engineering and Mathematics}

Tendo em mente todo o estereótipo em STEM que mulheres são enquadradas, esperávamos que sua baixa representatividade também representaria também baixa contribuição. Mais ainda, esperávamos que o estereótipo de mulheres em tecnologia~\cite{hyde1990gender} e aquele em que mulheres não possuem um taleto inato para ciência~\cite{} fosse influenciar na opinião que os outros usuários teriam das suas colegas, fazendo com que mulheres recebessem feedback negativo nas suas contribuições, indiscriminadamente, levando o conteúdo das mulheres a serem classificados com menor qualidade pela comunidade. Sem esquecer de que outros estudos já tinham identificado o conteúdo vindo de mulheres como de qualidade inferior comparado ao de homens, tanto em tarefas de matemática~\cite{campbell1986effects} quanto em comunidades online~\cite{collier2012conflict}.

Contudo, não é isso que acontece. Mulheres produzem conteúdo de tão boa qualidade quanto homens, segundo a visão da comunidade. Vemos que estas mulheres contribuem bastante com perguntas e comentários, tendo uma frequência de contribuições muitas vezes maiores que a de homens. O que achamos suspeito foi a menor proporção de comunidades onde mulheres respondem mais e principalmente estas estarem relacionadas à assuntos de STEM; e em praticamente todas as comunidades que observamos diferença estatística quanto ao tempo de vida, homens passavam mais tempo na comunidade.

Nosso resultado se alinha com o de ~\cite{yang2010activity}, que apesar de não estudar a mesma comunidade de perguntas e resposta, encontrou que usuários que têm o tempo de vida maior na comunidade preferem responder à perguntar. No nosso estudo, verificamos que estes usuários são geralmente homens, assim como encontrou ~\cite{Vasilescu27092013}. Vasilescu et al considera este resultado insalubre para a comunidade, visto que bons contribuidores estão deixando a comunidade. Nós acreditamos que exista uma barreira para que mulheres continue contribuindo, podendo ser desde a aquisição do pensamento de que não são capazes de contribuir por parte delas, até por perseguições/assédios dentro da própria comunidade por parte dos outros usuários. Estas hipóteses, contudo, pedem uma investigação mais detalhada, fora do escopo desta pesquisa.

Quanto ao tempo de vida de mulheres nestas comunidades, nota-se que a diferença de tempo é maior em comunidades mais antigas, vide tabela \ref{table:lifetime}. Isso pode se dar por que tais comunidades, a exemplo do StackOverflow, foram criadas por um grupo de homens e se mantiveram assim por bastante tempo, até se tornarem popular. Este grupo de usuários provavelmente tem um tempo de vida bem maior do que as mulheres que eventualmente ingressaram na comuindade. 

% "StackOverflow was founded by men, and this community may continued as a group of men until reach a certain popularity, number of users and before the birth of StackExchange along with other communities, as most of them come from branches of StackOverflow. "

Apesar de não ser relacionada com sua real performance, a percepção que uma pessoa tem de sua capacidade cognitiva em um campo influencia como esta pessoa acha que se sairá em uma determinada tarefa. Mais ainda, achar que não vai se dar bem em uma tarefa pode fazer com que, caso seja dada a opção, a pessoa nem chegue a realizar tal atividade. Mulheres acham que não são boas em tarefas sobre ciência e isso às influencia a não participar do campo~\cite{ehrlinger2003chronic}. Isso pode explicar porque a proporção de mulheres é tão baixa nas comunidades sobre STEM, mas elas se dão tão bem quanto homens na maioria delas.

% "This pattern of data suggests, by extension, that women might disproportionately avoid scientific pursuits because their self-views lead them to mischaracterize how well they are objectively doing on any given scientific task. Because they think they are doing more poorly than do men, they are more likely than men to avoid science when given an option." Texto de ehrlinger2003chronic

Uma justificativa para mulheres responderem menos em comunidades de STEM pode ser a de que respostas podem ser consideradas mais competitivas do que perguntas. Afinal, todas as respostas passam por um "processo seletivo" para ser a melhor resposta de uma determinada pergunta, o que não acontece com perguntas nem comentários. Já se sabe que mulheres não se dão bem em ambientes competitivos, principalmente quando precisam competir com homens~\cite{gneezy2003performance}, e tendem a evitá-los~\cite{niederle2005women,croson2009gender}. Além de que, demonstrar conhecimento em um assunto desta área pode exigir uma auto-confiança no campo que mulheres não têm.

Todavia, se olharmos para comunidades de áreas de STEM que são conhecidas por terem mais mulheres, observamos que o resultado é similar aos de comunidades não relacionadas à STEM. Tomando como exemplo UX (user experience) mulheres respondem mais nesta comunidade, o que pode ser uma demonstração de auto confiança no seu conhecimento no assunto. Mais ainda, UX não é considerada uma disciplina que necessita de um conhecimento inato para ter sucesso na área, sendo este mais um motivo para mulheres não se inibirem em participar desta comunidade de todas as formas possíveis. 

Por fim, ver que mulheres têm aumentado sua proporção de contribuições ao longo do tempo, principalmente em comunidades sobre assuntos relacionados à STEM, nos dá uma esperança que a barreira que existe para que mulheres entrem e contribuam para estas comunidades esteja diminuindo. Já que outras comunidades onde mulheres que já estão envolvidas tendem a continuar fazendo parte dela. Isto pode ser encontrado em comunidades FOSS~\cite{powell2010gender}. Uma maneira de aumentar a auto confiança destas mulheres quanto ao seu conhecimento também é essencial para aumentar estas proporções, já que elas mesmas consideram ter auto confiança como essencial para se manter em comunidades que elas são maioria~\cite{powell2010gender}.


% Esse parágrafo é interessante, mas não acho que cabe aqui.
% However, in Wikipedia, it has been found that women are more likely to be involved in social- and community-oriented areas of the site~\cite{lam2011wp}. This is in line with women being more likely to comment in the communities we study, where comments are the primarily means of socialization. Nevertheless, the higher likelihood of women to produce questions in three of the communities remains an open question for future work. A possible explanation to be considered is that self-selection leads more men contribute as answering-only users in these communities.


\begin{table}[h]
\centering
\begin{tabular}{@{}cccc@{}}
\toprule
{\small\textit{Median Diff.}} & {\small \textit{Category}} & {\small \textit{Community}} & {\small \textit{Age  }} \\ \midrule
-346.87                            & technology         & stackoverflow      & 76                 \\ \midrule
-327.76                           & science            & mathoverflow       & 62                 \\ \midrule
-70.93                            & technology         & superuser          & 64                 \\ \midrule
-35.29                            & technology         & serverfault        & 67                 \\ \midrule
-17.15                            & technology         & dsp                & 39                 \\ \midrule
-14.50                            & technology         & apple              & 51                 \\ \midrule
-12.62                            & life-arts          & diy                & 52                 \\ \midrule
-9.05                             & culture-recreation & english            & 51                 \\ \midrule
-8.80                             & life-arts          & scifi              & 46                 \\ \midrule
-7.26                             & technology         & networkengineering & 18                 \\ \midrule
-6.42                             & culture-recreation & mechanics          & 44                 \\ \midrule
-4.98                             & life-arts          & photo              & 52                 \\ \midrule
-1.92                             & culture-recreation & travel             & 41                 \\ \midrule
-1.66                             & science            & philosophy         & 41                 \\ \midrule
\textbf{74.72}                             & technology         & wordpress          & 51                 \\ \bottomrule
\end{tabular}
\caption{Table captions should be placed below the table. We recommend table lines be 1 point, 25\% black. Minimize use of unnecessary table lines.}~\label{table:lifetime}
\end{table}
% TODO: Caption!!



\subsection{Discussão do design}

O que nosso estudo deixa bem claro é que as comunidades do StackExchange possuem duas barreiras para a contribuição por parte das mulheres: um para que elas entrem e comecem a participar da comunidade e outra para que elas permaneçam tanto tempo quanto seus colegas do sexo masculino. As comunidades do StackExchange pode estar perdendo muito, em termos de conteúdo, com essa postura por parte das mulheres em seu sites.

É provável que exista preconceito e perseguição dentro destas comunidades sim, assim como boa parte das comunidades onlines. Identificar e eliminar usuários que não têm um comportamento de acordo com os termos da comunidade é uma boa prática. Contudo, é necessário mostrar para as mulheres que isto está sendo feito, para que elas possam se sentir seguras neste ambiente. 

Mais ainda, uma campanha feminista (lembrando que o feminismo promove a igualdade entre os sexos e não a dominação do sexo feminino) para promover o site pode ser um bom primeiro contato com a comunidade feminina e procurar saber diretamente delas como melhorar.

% subsection sugestoes (end)


% Note to self: estou começando a achar que existe prejudice e harassment sim, mas as mulheres que conseguem se engajar com estas comunidades de tecnologia conseguem ignorar essas negatividades. Newcomers podem ver isto como barreira e nem se quer entrar na comunidade.

