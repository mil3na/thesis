%!TEX root = main_mestrado.tex
\chapter{Resultados}
\label{ch:resultados}

\begin{figure}
  \includegraphics[width=1\columnwidth]{figures/perguntas_mulheres.pdf}
  \caption[Comparação do número de perguntas]{Comparação do número de perguntas provenientes de contribuidores de diferentes gêneros nos sites onde pode-se verificar uma diferença estatisticamente significante e mulheres contribuem mais do que homens.}~\label{fig:questions}
\end{figure}

Este capítulo apresenta os resultados para cada uma das quatro perguntas de pesquisa. Quando descrevemos as diferenças para uma determinada variável de um conjunto de sites, nós utilizamos da notação $M/S/H$, em que $M$ é o número de sites onde mulheres obtiveram valores para esta variável significativamente maiores, $S$ o número de sites onde não obteve-se diferença significativa e, por fim, $H$ é o número de sites onde homens obtiveram valores significativamente maiores. Mais ainda, nas figuras aqui apresentadas, as caixas representam os quartis das distribuições enquanto os traços mostram o primeiro e nonagésimo nono percentis. Para visualizar a distribuição em escala de log, acrescentamos 1 ao valor original do número de contribuições realizadas por cada usuário. Por fim, lembramos que o teste MWW pode achar diferença estatística na soma dos rankings mesmo que os quartis sejam similares nas duas amostras.

\section{Número de contribuições}
Para todos os tipos de contribuições considerados, um padrão em comum surgiu em nossos resultados: a diferença da quantidade de contribuições provenientes de homens e mulheres não é significativa na maioria dos sites (teste \emph{Mann-Whitney U}, $\alpha = 0.05$). E, mesmo para os sites onde pudemos observar diferença estatística, esta não foi grande. O estimador de Hodges-Lehmann nos indica que, para qualquer tipo de contribuição, não há diferença maior do que uma unidade de contribuição entre homens e mulheres.

Dos sites em que pudemos observar uma diferença estatística significativa, existe um número maior de comunidades nas quais podemos observar que mulheres são mais ativas que homens do que o contrário. Esta diferença é maior quando observamos o número de perguntas ($21/62/2$), seguido pelo número de respostas ($18/56/11$) e, finalmente, o número total de contribuições ($10/73/2$). 

Interessante notar que existe uma diferença no padrão de categorias nas quais podemos notar diferenças no número de respostas entre homens e mulheres, comparando aos outros tipos de contribuição. Para perguntas, comentários e a soma de todas as contribuições, as comunidades onde mulheres são geralmente mais ativas pertencem a categorias relacionadas à \emph{STEM}. Para perguntas a distribuição de comunidades relacionadas à \emph{STEM} é $16/28/0$, como pode ser observado na Figura~\ref{fig:questions}. Contudo, quando olhamos para respostas, apenas três das dezoito comunidades onde mulheres respondem mais do que homens são relacionadas a \emph{STEM} (como pode ser visto na Figura~\ref{fig:answers}). É mais comum encontrar comunidades das categorias \emph{Life/Arts} ou \emph{Professional} dentre os sites que pode-se observar que mulheres respondem significativamente mais do que homens. Mais ainda, duas das três comunidades relacionadas à \emph{STEM}, onde mulheres respondem mais que homens abordam assuntos que são estereotipados como mais femininos: \emph{UX} e \textit{Software Quality Assurance}. As comunidades onde homens respondem significativamente mais do que mulheres podem ser vistas na Figura~\ref{fig:answers-men}.

\begin{figure}
  \centering
  \includegraphics[width=1\columnwidth]{figures/respostas_mulheres.pdf}
  \caption[Comparação do número de respostas em comunidades onde mulheres respondem mais.]{Comparação do número de respostas provenientes de contribuidores de diferentes gêneros nos sites onde pode-se verificar uma diferença estatisticamente significante e mulheres contribuem mais do que homens.}~\label{fig:answers}
\end{figure}


Estes resultados nos mostram que mulheres que identificam seu gênero e que se engajam à comunidade, na maioria dos sites, não contribuem menos do que os homens. Ao contrário: quando podemos observar diferenças, essas frequentemente apontam que mulheres contribuem mais do que homens.

\section{Qualidade} % (fold)
\label{subsec:qualidade}
Poucos foram os sites que apresentaram uma diferença significante em como a comunidade avalia a qualidade das contribuições de homens e mulheres. Esta diferença ocorre menos frequentemente do que as diferenças em número de contribuições. Para a taxa de aceitação ($7/76/2$) e utilidade média das respostas ($9/74/2$) dos usuários, um número pequeno de comunidades tende a atribuir às respostas das mulheres uma qualidade maior do que às dos homens. Para a média dos votos de uma pergunta, podemos observar um equilíbrio nos sites que estudamos ($8/67/10$). Não encontramos um padrão relacionado às categorias a partir dos quais podemos encontrar diferença significativa na avaliação de qualidade. E, assim como para o número de contribuições, o estimador de Hodges-Lehmann nos indica um valor ínfimo na diferença de qualidade entre homens e mulheres, quando o teste estatístico indica que há uma diferença entre as amostras.

% section qualidade (end)


\section{Dedicação}% (fold)
Para as métricas relacionadas ao tempo dedicado à comunidade, as diferenças observadas possuem um padrão bem claro. Primeiramente, para a maioria dos sites, não há diferença estatística quanto ao tempo de vida ou frequência de atividade de homens e mulheres. Contudo, quando constatamos diferença no tempo de vida, estas acontecem porque homens passam mais tempo na comunidade do que mulheres ($1/70/14$). E o contrário acontece com relação à frequência de contribuição em dias ativos: mulheres contribuem mais frequentemente do que homens ($10/74/1$).

\begin{figure}[!b]
  \raggedleft
  \includegraphics[width=1\columnwidth]{figures/respostas_homens.pdf}
  \caption[Comparação do número de respostas em comunidades onde homens respondem mais.]{Comparação do número de respostas provenientes de contribuidores de diferentes gêneros nos sites onde pode-se verificar uma diferença estatisticamente significante e homens contribuem mais do que mulheres. }~\label{fig:answers-men}
\end{figure}

Nós também notamos que as maiores diferenças entre os gêneros quanto ao tempo de vida na comunidade ocorre no StackOverflow, o mais popular e também o mais antigo site do \emph{StackExchange}. Os próximos três sites em grandeza de diferença de tempo de vida também são antigos e relacionados a \emph{STEM}: \emph{MathOverflow}, \emph{SuperUser} e \emph{ServerFault}. Estas diferenças podem ser observadas na Tabela~\ref{table:lifetime}. 

\begin{table}[!b]
\centering
\begin{tabular}{@{}rllr@{}}
\toprule
{\small\textit{Dif. Mediana}} & {\small \textit{Categoria}} & {\small \textit{Comunidade}} & {\small \textit{Idade (meses) }} \\ \midrule
-346.87                            & technology         & stackoverflow      & 76                 \\ \midrule
-327.76                           & science            & mathoverflow       & 62                 \\ \midrule
-70.93                            & technology         & superuser          & 64                 \\ \midrule
-35.29                            & technology         & serverfault        & 67                 \\ \midrule
-17.15                            & technology         & dsp                & 39                 \\ \midrule
-14.50                            & technology         & apple              & 51                 \\ \midrule
-12.62                            & life-arts          & diy                & 52                 \\ \midrule
-9.05                             & culture-recr. & english            & 51                 \\ \midrule
-8.80                             & life-arts          & scifi              & 46                 \\ \midrule
-7.26                             & technology         & networkengineering & 18                 \\ \midrule
-6.42                             & culture-recre. & mechanics          & 44                 \\ \midrule
-4.98                             & life-arts          & photo              & 52                 \\ \midrule
-1.92                             & culture-recre. & travel             & 41                 \\ \midrule
-1.66                             & science            & philosophy         & 41                 \\ \midrule
\textbf{74.72}                             & technology         & wordpress          & 51                 \\ \bottomrule
\end{tabular}
\caption[Diferença de tempo de vida entre homens e mulheres]{Sites onde podemos observar uma diferença estatística significante entre homens e mulheres quanto ao seu tempo de vida na comunidade. Apresentamos a diferença entre a mediana de dias ativos, a idade da comunidade em meses e sua categoria.}~\label{table:lifetime}
\end{table}

Com relação à frequência, um padrão diferente pode ser observado. Em todos os sete sites relacionados a \emph{STEM} que apresentaram diferença estatística com relação à frequência de atividade, mulheres contribuem mais frequentemente do que homens. Assimilando isto com o resultado relacionado ao tempo de vida, podemos inferir que, apesar de as mulheres deixarem a comunidade antes dos homens, elas contribuem mais durante o tempo que ainda estão associadas ao site.


\section{Contribuições e novos registros ao longo do tempo}
Observando as contribuições de homens e mulheres ao longo do tempo, notamos que em, aproximadamente, um quarto dos sites estudados a proporção das contribuições provenientes de mulheres está aumentando ($20/60/5$). Apesar de não tão evidente, das comunidades que apresentaram diferença estatística quanto aos registros feitos por homens e mulheres ao longo do tempo, é possível ver um padrão: os registros de mulheres têm aumentado ($8/76/1$). Ambos resultados são mais visíveis se considerarmos apenas os sites relacionados à \emph{STEM}: $16/25/3$ para contribuições ao longo do tempo e $6/38/0$ para a proporção de novos registros.

\section{Focando nos maiores sites} 

Os três maiores sites do \emph{StackExchange} -- \emph{StackOverflow}, \emph{ServerFault} e \emph{SuperUser} -- são muito maiores e mais antigos que os demais sites disponíveis na nossa amostra. Juntamente com os sites \emph{Mathematics} e \emph{Programmers} eles possuem uma proporção significante dos usuários de toda a plataforma \emph{StackExchange}, sendo os cinco maiores sites, em número de usuários, da nossa amostra, achamos por bem evidenciar o comportamento dos usuários nestas comunidades. 

No \emph{StackOverflow}, primeira comunidade da plataforma, mulheres perguntam mais e homens costumam responder mais, apesar de não encontrarmos diferença estatística no total de contribuições feita por ambos os gêneros. Perguntas e respostas vistas como mais qualificadas pelos usuários advêm de homens. Ademais, o padrão de homens possuírem maior tempo de vida na comunidade, enquanto mulheres possuem maior frequência de contribuição também pode ser visto neste site. E, ainda, a proporção de mulheres se cadastrando e contribuindo para a comunidade tem aumentado nos últimos anos.

Os três maiores sites do \emph{StackExchange} -- \emph{StackOverflow}, \emph{ServerFault} e \emph{SuperUser} -- são muito maiores e mais antigos que os demais sites disponíveis na nossa amostra. Juntamente com os sites \emph{Mathematics} e \emph{Programmers} eles possuem uma proporção significante dos usuários de toda a plataforma \emph{StackExchange}, sendo os cinco maiores sites, em número de usuários, da nossa amostra, achamos por bem evidenciar o comportamento dos usuários nestas comunidades. 

Em todas as demais comunidades dentre as cinco maiores comunidades da nossa amostra, homens respondem mais que mulheres. Tal fato já era esperado, já que todas tratam de assuntos relacionados à \emph{STEM}. Mulheres perguntam mais nos sites \emph{Programmers} e \emph{Mathematics}. E, nesse último, mulheres comentam e têm um numero total de contribuições maior que o dos homens também. Quanto à qualidade, apenas a \emph{ServerFault} apresentou diferença estatística e em apenas uma métrica (média de votos das perguntas), favorável aos homens. Nos sites \emph{ServerFault}, \emph{SuperUser}, homens apresentam tempo de vida na comunidade maior do que as mulheres, enquanto apenas na \emph{Mathematics} mulheres possuem uma frequência de postagem maior que a dos homens. As demais não apresentam diferença estatística. Ao longo do tempo, mulheres têm contribuído mais do que homens nos sites \emph{Programmers}, \emph{Mathematics} e \emph{ServerFault}. 

Por fim, uma comparação do número de contribuições nos cinco maiores sites do \emph{StackExchange} relacionados a \emph{STEM} pode ser encontrada na Figura~\ref{fig:top-five}.


\begin{figure}[!b]
  \centering
  \includegraphics[height=0.65\paperheight]{figures/top5_quantidade.pdf}
  \caption[Comparação do número de contribuições na 5 maiores comunidades.]{Comparação do número de contribuições provenientes de usuários de diferentes gêneros nos cinco maiores sites (em quantidade de usuários) relacionados a \emph{STEM} do \emph{StackExchange}.}~\label{fig:top-five}
\end{figure}


% \section{Resultados por categoria}



% section results (end)

